\documentclass[a4paper,11pt]{article}
\usepackage{pdflscape}
\usepackage[utf8]{inputenc}
\usepackage[T1]{fontenc}
%\usepackage{fourier} % math & rm
%\usepackage{amsthm,amsfonts,amsmath,amssymb,textcomp}
\usepackage{pst-all,pstricks-add,pst-eucl}
\everymath{\displaystyle}
\usepackage{fp,ifthen}
%\usepackage{color}
%\usepackage{graphicx}
\usepackage{setspace}
\usepackage{array}
\usepackage{tabularx}
\usepackage{supertabular}
\usepackage{hhline}
\usepackage{variations}
\usepackage{enumerate}
\usepackage{pifont}
\usepackage{framed}
\usepackage[fleqn]{amsmath}
\usepackage{amssymb}
\usepackage[framed]{ntheorem}
\usepackage{multicol}
\usepackage{kpfonts}
\usepackage{manfnt}
\usepackage{hyperref}

%\usepackage[hmargin=2.5cm, vmargin=2.5cm]{geometry}
\usepackage{vmargin}          % Pour fixer les marges du document
\setmarginsrb
{1.5cm} 	%marge gauche
{0.5cm} 	  %marge en haut
{1.5cm}     %marge droite
{0.5cm}   %marge en bas
{1cm} 	%hauteur de l'entête
{0.5cm}   %distance entre l'entête et le texte
{1cm} 	  %hauteur du pied de page
{0.5cm}     %distance entre le texte et le pied de page

\newcommand{\R}{\mathbb{R}}
\newcommand{\N}{\mathbb{N}}
%\newcommand{\D}{\mathbb{D}}
\newcommand{\Z}{\mathbb{Z}}
\newcommand{\Q}{\mathbb{Q}}
\newcommand{\C}{\mathbb{C}}
\newcommand{\e}{\text{e}}
\newcommand{\dx}{\text{d}x}
\newcommand{\vect}[1]{\mathchoice%
  {\overrightarrow{\displaystyle\mathstrut#1\,\,}}%
  {\overrightarrow{\textstyle\mathstrut#1\,\,}}%
  {\overrightarrow{\scriptstyle\mathstrut#1\,\,}}%
  {\overrightarrow{\scriptscriptstyle\mathstrut#1\,\,}}}
\newcommand\arraybslash{\let\\\@arraycr}
\renewcommand{\theenumi}{\textbf{\arabic{enumi}}}
\renewcommand{\labelenumi}{\textbf{\theenumi.}}
\renewcommand{\theenumii}{\textbf{\alph{enumii}}}
\renewcommand{\labelenumii}{\textbf{\theenumii.}}
\renewcommand{\and}{\wedge}

\theoremstyle{break}
\theorembodyfont{\upshape}
\newframedtheorem{Theo}{Théorème}
\newframedtheorem{Prop}{Propriété}
\newframedtheorem{Def}{Définition}

\newtheorem{Term}{Terminologie}
\newtheorem{Rq}{Remarque}
\newtheorem{Ex}{Exemple}
%\newtheorem{exo}{Exercice}

%\theorembodyfont{\small \sffamily}
%\newtheorem{sol}{solution}

\newenvironment{sol}% 
{\def\FrameCommand{\hspace{0.5cm} {\color{black} \vrule width 1pt} \hspace{-0.7cm}}%
  \framed {\advance\hsize-\width}
  \noindent \small \sffamily  %\underline{Solution :}%\\
}%
{\endframed}

\newrgbcolor{vert}{0 0.4 0}
\newrgbcolor{bistre}{1 .50 .30}
\setlength\tabcolsep{1mm}
\renewcommand\arraystretch{1.3}

\everymath{\displaystyle}
\hyphenpenalty 10000 %supprime toutes les césures
%\setcounter{secnumdepth}{0}
%\newcounter{saveenum}

\usepackage[frenchb]{babel}
\usepackage{fancyhdr,lastpage}
\usepackage{fancybox}

%\headheight 15.0 pt
\fancyhead[L]{}
\fancyhead[C]{Cahier de textes première SD}
\fancyhead[R]{Année 2015-2016}
\fancyfoot[L]{{\scriptsize\textsl{Thomas Gire Cité scolaire de Lorgues}}}
%\fancyfoot[C]{\scriptsize\thepage}
%\fancyfoot[C]{\scriptsize\thepage/\pageref{LastPage}}

\title{}
\author{}
\date{}

%\pagestyle{empty}
\pagestyle{fancy}
\usepackage[np]{numprint}

\renewcommand\arraystretch{1.8}

\newcounter{numero}
\newcommand{\exo}{
  \addtocounter{numero}{1}%
  \textbf{\underline{Exercice \arabic{numero}:}}\quad}

\frenchbsetup{StandardEnumerateEnv=true}
\usepackage{etex}
\usepackage{tikz,tkz-tab}

\newframedtheorem{Dev}{Devoirs}
\renewcommand{\theDev}{\empty{}} 

\newcommand{\dm}{
  \textbf{\underline{Devoir à la maison:}}\quad \vspace{0.5cm}}

\begin{document}
  \setlength{\unitlength}{1mm}
  \setlength\parindent{0mm}
  
  
  %\exo
  ~
  \medskip
  
  Jeudi 3 septembre 2015
   \begin{enumerate}
   %\item Ecrire les devoirs pour 04/09/2015: Variations et nombre de racines: Exercices 1 et 2.
   \item \'Equations et inéquations à résoudre de tête: 
   \href{https://github.com/mathlorgues/math1sd1516/blob/master/20150903/33-34-35p38.jpg}
   {Exercices 33-34-35 p 38}
   \item Variations d'un trinôme du second degré.
   \begin{enumerate}
    \item 
    \href{https://github.com/mathlorgues/math1sd1516/blob/master/20150903/definitions.pdf}
    {Définition 1. 2.}) 
    Variations et terminologie extremum.
    \item 
    \href{https://github.com/mathlorgues/math1sd1516/blob/master/20150903/activite.pdf}
    {Activité 1})
    \item 
    \href{https://github.com/mathlorgues/math1sd1516/blob/master/20150903/propositions.pdf}
    {Théorème 1})
  (révision de la classe de seconde)
   \end{enumerate}
  \end{enumerate}
  
  \begin{Dev}
    
    Apprendre à utiliser le théorème 1 (révision classe de seconde):
   \begin{enumerate}
     \item Savoir réaliser le tableau de variations d'un trinôme du second degré.
     \item Savoir calculer les coordonnées du sommet d'une parabole.
  \end{enumerate}
 % \vspace{0.5cm}
  \begin{exo}[Pour vendredi 4 septembre]
    
    Dresser le tableau de variations des fonctions définies sur $\mathbb{R}$ 
    suivantes et donner les coordonnées du sommet de leur graphe : 
    \begin{enumerate}
      \item $f(x)=3 x^2-12 x +13$.
      \item $g(x)=-2x^2-4x+1$.
      \item $h(x)=4(x-1)^2+2$.
      \item $i(x)=2(x+2)^2-2$.
    \end{enumerate}
    
  \end{exo}
  
  \end{Dev}
  
  Vendredi 4 septembre 2015
  \begin{enumerate}
   %\item Ecrire les devoirs pour 04/09/2015: Variations et nombre de racines: Exercices 1 et 2.
   \item Factorisations utilisant une identité remarquable: .
   \href{https://github.com/mathlorgues/math1sd1516/blob/master/20150904/1-2-3p22.png}
   {ex 1 p 22}
   \item Variations d'un trinôme du second degré:
   \href{https://github.com/mathlorgues/math1sd1516/blob/master/20150904/activite.pdf}
   {Activité 2}
    1. 2.
   
  \end{enumerate}
  
   \begin{Dev}
     
     Savoir exploiter l'identité remarquable $a^2-b^2=(a-b)(a+b)$ pour factoriser 
     une expression littérale sans facteur apparent.
     
     \begin{exo}[Pour lundi 7 septembre]
       
    Factoriser les expressions suivantes:
    \begin{enumerate}
     \item $f(x)=x^2-1$
     \item $g(x)=(2x)^2-9$
     \item $h(x)=4x^2-16$
     \item $i(x)=x^2-2$
     \item $j(x)=3x^2-5$
     \item $k(x)=x^2+1$
    \end{enumerate}
    
     
    \end{exo}

  
  \end{Dev}
  
  \newpage
  
  Lundi 7 septembre 2015
  \begin{enumerate}
    
    \item \'Etude du nombre d'intersections d'une parabole avec l'axe des abscisses.
    \begin{enumerate}
      \item 
      \href{https://github.com/mathlorgues/math1sd1516/blob/master/20150907/activite.pdf}
   {Activité 2} 3.-7.
      \item 
      \href{https://github.com/mathlorgues/math1sd1516/blob/master/20150907/propositions.pdf}
   {Proposition 1} : Positions de paraboles.
      \item 
      \href{https://github.com/mathlorgues/math1sd1516/blob/master/20150907/definitions.pdf}
   {Terminologie 1} : Racines d'un trinôme.
      \item 
      \href{https://github.com/mathlorgues/math1sd1516/blob/master/20150907/propositions.pdf}
   {Proposition 2} : Reformulation de la proposition 1 en terme de racines.
    \end{enumerate}
    \item Résolution d'équations du second degré exigibles en classe de seconde 
    \href{https://github.com/mathlorgues/math1sd1516/blob/master/20150907/46p38.jpg}
   {Ex 46 p 38}.
    \item 
    \href{https://github.com/mathlorgues/math1sd1516/blob/master/20150907/38-42p38.jpg}
   {Ex 38 à 42 p 38}.
    
  \end{enumerate}
  
  \begin{Dev*}
    
    \begin{enumerate}
      
      \item Comprendre la proposition 1.
      \begin{enumerate}
	\item Savoir déterminer la nature de l'extremum sur une parabole.
	\item Exploiter sa valeur pour déduire le nombre d'intersections 
	avec l'axe des abscisses.
	
      \end{enumerate}
      \item Apprendre terminologie 1 : Racines d'un trinôme du second degré.
      \vspace{0.5cm}
      
      \begin{dm}[À rendre avant le lundi 14 septembre]
	
	Déterminer le nombre d'intersections avec l'axe des abscisses pour chaque parabole
	suivante:
	\begin{enumerate}
	  \item $\mathcal{P}_1:y=2(x+2)^2+1$
	  \item $\mathcal{P}_2:y=-x^2+6x-7$.%-(x-3)^2+2
	  \item $\mathcal{P}_3:y=x^2+2x+1$.
	  \item $\mathcal{P}_4:y=-2(x+2)^2$.
	\end{enumerate} 
      \end{dm}
      
      
      
    \end{enumerate}
  \end{Dev*}
  
  
  Mardi 8 septembre 2015
   \begin{enumerate}   
    \item Discriminant et racines d'un trinômes.
    \begin{enumerate}
      \item 
      \href{https://github.com/mathlorgues/math1sd1516/blob/master/20150908/definitions.pdf}
   {Définition 1}: Discriminant d'un trinôme.
      \item 
      \href{https://github.com/mathlorgues/math1sd1516/blob/master/20150908/activite.pdf}
   {Activité 3} 1. 2. 3. : Nombre de racines.
      \item 
      \href{https://github.com/mathlorgues/math1sd1516/blob/master/20150908/propositions.pdf}
   {Théorème 1}(central): Calcul des racines et factorisations.
    \end{enumerate}
    \item  
    \href{https://github.com/mathlorgues/math1sd1516/blob/master/images/47-52p38.png}
   {Ex 47 à 52 p 38} : 
    Résolution d'équations du second degré
    \item 
    \href{https://github.com/mathlorgues/math1sd1516/blob/master/images/53-54%20p38.png}
   {Ex 53,54 p 38} : 
    Factorisation d'un trinôme du second degré.
  \end{enumerate}
  
  \begin{Dev}[pour jeudi 10 semptembre 2015]
    
    \begin{enumerate}
      
      \item Apprendre la formule du discriminant.
      \item Apprendre le théorème central sur le calcul des racines 
	d'un trinôme du second degré.
      \begin{enumerate}
	\item Savoir calculer le discriminant d'un trinôme du second degré. 1
	\item Savoir utiliser le signe du discriminant pour déterminer
	le nombre de racines.
	\item Savoir calculer les racines éventuelles.
      \end{enumerate}
      \item
      \href{https://github.com/mathlorgues/math1sd1516/blob/master/images/55-56p38.png}
   {Ex 55,56 p 38} : \'Equations du second degré avec un paramètre.
      
    \end{enumerate}
  \end{Dev}


  
\end{document}
