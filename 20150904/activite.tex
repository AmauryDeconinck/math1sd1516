\documentclass[a4paper,11pt]{article}
\usepackage{pdflscape}
\usepackage[utf8]{inputenc}
\usepackage[T1]{fontenc}
%\usepackage{fourier} % math & rm
%\usepackage{amsthm,amsfonts,amsmath,amssymb,textcomp}
\usepackage{pst-all,pstricks-add,pst-eucl}
\everymath{\displaystyle}
\usepackage{fp,ifthen}
%\usepackage{color}
%\usepackage{graphicx}
\usepackage{setspace}
\usepackage{array}
\usepackage{tabularx}
\usepackage{supertabular}
\usepackage{hhline}
\usepackage{variations}
\usepackage{enumerate}
\usepackage{pifont}
\usepackage{framed}
\usepackage[fleqn]{amsmath}
\usepackage{amssymb}
\usepackage[framed]{ntheorem}
\usepackage{multicol}
\usepackage{kpfonts}
\usepackage{manfnt}

%\usepackage[hmargin=2.5cm, vmargin=2.5cm]{geometry}
\usepackage{vmargin}          % Pour fixer les marges du document
\setmarginsrb
{1.5cm} 	%marge gauche
{0.5cm} 	  %marge en haut
{1.5cm}     %marge droite
{0.5cm}   %marge en bas
{1cm} 	%hauteur de l'entête
{0.5cm}   %distance entre l'entête et le texte
{1cm} 	  %hauteur du pied de page
{0.5cm}     %distance entre le texte et le pied de page

\newcommand{\R}{\mathbb{R}}
\newcommand{\N}{\mathbb{N}}
%\newcommand{\D}{\mathbb{D}}
\newcommand{\Z}{\mathbb{Z}}
\newcommand{\Q}{\mathbb{Q}}
\newcommand{\C}{\mathbb{C}}
\newcommand{\e}{\text{e}}
\newcommand{\dx}{\text{d}x}
\newcommand{\vect}[1]{\mathchoice%
  {\overrightarrow{\displaystyle\mathstrut#1\,\,}}%
  {\overrightarrow{\textstyle\mathstrut#1\,\,}}%
  {\overrightarrow{\scriptstyle\mathstrut#1\,\,}}%
  {\overrightarrow{\scriptscriptstyle\mathstrut#1\,\,}}}
\newcommand\arraybslash{\let\\\@arraycr}
\renewcommand{\theenumi}{\textbf{\arabic{enumi}}}
\renewcommand{\labelenumi}{\textbf{\theenumi.}}
\renewcommand{\theenumii}{\textbf{\alph{enumii}}}
\renewcommand{\labelenumii}{\textbf{\theenumii.}}

\theoremstyle{break}
\theorembodyfont{\upshape}
\newframedtheorem{Theo}{Théorème}
\newframedtheorem{Prop}{Propriété}
\newframedtheorem{Def}{Définition}

\newtheorem{Rq}{Remarque}
\newtheorem{Ex}{Exemple}
%\newtheorem{exo}{Exercice}

%\theorembodyfont{\small \sffamily}
%\newtheorem{sol}{solution}

\newenvironment{sol}% 
{\def\FrameCommand{\hspace{0.5cm} {\color{black} \vrule width 1pt} \hspace{-0.7cm}}%
  \framed {\advance\hsize-\width}
  \noindent \small \sffamily  %\underline{Solution :}%\\
}%
{\endframed}

\newrgbcolor{vert}{0 0.4 0}
\newrgbcolor{bistre}{1 .50 .30}
\setlength\tabcolsep{1mm}
\renewcommand\arraystretch{1.3}

\everymath{\displaystyle}
\hyphenpenalty 10000 %supprime toutes les césures
%\setcounter{secnumdepth}{0}
%\newcounter{saveenum}

\usepackage[frenchb]{babel}
\usepackage{fancyhdr,lastpage}
\usepackage{fancybox}

%\headheight 15.0 pt
\fancyhead[L]{Second degré}
\fancyhead[C]{ Activité n° 1 Paraboles}
\fancyhead[R]{ \today}
\fancyfoot[L]{{\scriptsize\textsl{ Thomas Gire Cité scolaire de Lorgues}}}
%\fancyfoot[C]{\scriptsize\thepage}
%\fancyfoot[C]{\scriptsize\thepage/\pageref{LastPage}}

\title{Devoi}
\author{B Gault}
\date{2012-2013}

%\pagestyle{empty}
\pagestyle{fancy}
\usepackage[np]{numprint}

\renewcommand\arraystretch{1.8}

\newcounter{numero}
\newcommand{\exo}{
  \addtocounter{numero}{1}%
  \textbf{\underline{Exercice \arabic{numero}:}}\quad}

\frenchbsetup{StandardEnumerateEnv=true}
\usepackage{etex}
\usepackage{tikz,tkz-tab}

\begin{document}
  \setlength{\unitlength}{1mm}
  \setlength\parindent{0mm}
  
  
  %\exo
  ~
  \medskip
  
  \begin{enumerate}
   
    
    \item Dresser le tableau de variations des fonctions définies sur $\mathbb{R}$ 
    suivantes par : 
    \begin{enumerate}
      \item $e(x)=2x^2+8x-6$.
      \item $f(x)=-3x^2+3$.
      \item $g(x)=-3(x-1)^2+5$.
    \end{enumerate}
    
    \item En déduire, pour chacune de ces fonctions, la présence d'un extremum dont on 
    précisera la nature.
    \item Tracer sur la calculatrice la parabole d'équation $\mathcal{P}_1:y=(x+1)^2-1$.
    
    \begin{enumerate}
      
      \item Donner l'équation d'un axe de symétrie pour cette parabole.
      \item Où se situe le sommet de $\mathcal{P}_1$ par rapport à l'axe des abscisses ?
      \item Ce sommet correspond-il à un maximum ou à un minimum ?
      \item Cet extremum est-il positif ou négatif ?
      \item $\mathcal{P}_1$ coupe-t-elle l'axe des abscisses ?
      
    \end{enumerate}
    
    \item Même question pour les paraboles d'équations 
    $$\mathcal{P}_2:y=-2(x-1)^2-1 ,
    \mathcal{P}_3:y=(x+1)^2+1,
    \mathcal{P}_4:y=-2(x-1)^2+1$$
    \item Proposer un critère permettant de déterminer si une parabole coupe l'axe
    des abscisses en deux points $A_1(x_1,0)$ et $A_2(x_2,0)$.
    \item Quelles sont les coordonnées des sommets des paraboles suivantes ?
    $$\mathcal{P}_5:y=-4(x-2)^2 ,
    \mathcal{P}_6:y=(x^2+2x+1)^2$$
 
    \item Quelle est la position relative de la parabole et de l'axe des abscisses ?
    
  
  \end{enumerate}
  \vspace{0.5cm}
  


 
  
  
\end{document}
